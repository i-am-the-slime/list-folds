\begin{frame}
\frametitle{foldl}
The \lstinline[basicstyle=\ttfamily]$foldr$ function accepts three values:
\begin{enumerate}
\item \lstinline[basicstyle=\ttfamily]$f :: a -> b -> b$
\item \lstinline[basicstyle=\ttfamily]$z :: b$
\item \lstinline[basicstyle=\ttfamily]$list :: List a$
\end{enumerate}
to get back a value of the type \lstinline[basicstyle=\ttfamily]$b$.

\hrulefill

\lstinline[basicstyle=\ttfamily]$foldr :: (a -> b -> b) -> b -> List a -> b$
\lstinline[basicstyle=\ttfamily]$B FoldRight<A, B>(Func<A, B, B>, B, List<A>)$
\end{frame}

\begin{frame}
\frametitle{foldl}
\begin{block}{?}
\begin{center}
How does \lstinline[basicstyle=\ttfamily]$foldr$ take three values to that return value?
\end{center}
\end{block}
\end{frame}

\begin{frame}
\frametitle{foldr}
\begin{block}{constructor replacement}
The \lstinline[basicstyle=\ttfamily]$foldr$ function performs \textbf{constructor replacement}.
\end{block}
The expression \lstinline[basicstyle=\ttfamily]$foldr f z list$ replaces in \lstinline[basicstyle=\ttfamily]$list$:
\begin{itemize}
\item Every occurrence of \lstinline{Cons} \lstinline[basicstyle=\ttfamily]$(:)$ with \lstinline[basicstyle=\ttfamily]$f$.
\item Any occurrence of \lstinline{Nil} \lstinline[basicstyle=\ttfamily]$[]$ with \lstinline[basicstyle=\ttfamily]$z$\footnote{The \lstinline{Nil} constructor may be absent \textemdash an infinite list}.
\end{itemize}
\end{frame}

\begin{frame}
\frametitle{foldr}
\begin{block}{constructor replacement?}
\small
\begin{itemize}
\item suppose \lstinline[basicstyle=\ttfamily]$list = Cons A (Cons B (Cons C (Cons D Nil)))$
\item the expression \lstinline[basicstyle=\ttfamily, mathescape]!foldr f z list!
\item produces \lstinline[basicstyle=\ttfamily]$f A (f B (f C (f D z)))$
\end{itemize}
\end{block}
\end{frame}

\begin{frame}[fragile]
\frametitle{foldr}
\begin{block}{right folds replace constructors}
Let's multiply the integers of a list
\end{block}
\end{frame}

\begin{frame}[fragile]
\frametitle{foldr}
\begin{block}{multiply the integers of a list}
Supposing 
\begin{lstlisting}[style=haskell,basicstyle=\scriptsize\ttfamily,mathescape]
list = Cons 4 (Cons 5 (Cons 6 (Cons 7 Nil)))
\end{lstlisting}
\end{block}
\end{frame}

\begin{frame}[fragile]
\frametitle{foldr}
\begin{block}{multiply the integers of a list}
Supposing 
\begin{lstlisting}[style=haskell,basicstyle=\scriptsize\ttfamily,mathescape]
list = `Cons` 4 (`Cons` 5 (`Cons` 6 (`Cons` 7 `Nil`)))
\end{lstlisting}
\end{block}
\begin{center}
\LARGE
?
\end{center}
\end{frame}

\begin{frame}[fragile]
\frametitle{foldr}
\begin{block}{multiply the integers of a list}
\begin{itemize}
\item let \lstinline{`Cons` = (*)}
\item let \lstinline{`Nil`  = 1}
\end{itemize}
\end{block}
\end{frame}

\begin{frame}[fragile]
\frametitle{foldr}
\begin{block}{multiply the integers of a list}
Supposing 
\begin{lstlisting}[style=haskell,basicstyle=\scriptsize\ttfamily,mathescape]
list = `(*)` 4 (`(*)` 5 (`(*)` 6 (`(*)` 7 `1`)))
\end{lstlisting}
\end{block}
\begin{lstlisting}[style=haskell,basicstyle=\scriptsize\ttfamily,mathescape]
product list = foldr (*) 1 list
product = foldr (*) 1
\end{lstlisting}
\end{frame}

%

\begin{frame}[fragile]
\frametitle{foldr}
\begin{block}{right folds replace constructors}
Let's and \lstinline{(&&)} the booleans of a list
\end{block}
\end{frame}

\begin{frame}[fragile]
\frametitle{foldr}
\begin{block}{and \lstinline{(&&)} the booleans of a list}
Supposing 
\begin{lstlisting}[style=haskell,basicstyle=\tiny\ttfamily,mathescape]
list = Cons True (Cons True (Cons False (Cons True Nil)))
\end{lstlisting}
\end{block}
\end{frame}

\begin{frame}[fragile]
\frametitle{foldr}
\begin{block}{and \lstinline{(&&)} the booleans of a list}
Supposing 
\begin{lstlisting}[style=haskell,basicstyle=\tiny\ttfamily,mathescape]
list = `Cons` True (`Cons` True (`Cons` False (`Cons` True `Nil`)))
\end{lstlisting}
\end{block}
\begin{center}
\LARGE
?
\end{center}
\end{frame}

\begin{frame}[fragile]
\frametitle{foldr}
\begin{block}{and \lstinline{(&&)} the booleans of a list}
\begin{itemize}
\item let \lstinline{`Cons` = (&&)}
\item let \lstinline{`Nil`  = True}
\end{itemize}
\end{block}
\end{frame}

\begin{frame}[fragile]
\frametitle{foldr}
\begin{block}{and \lstinline{(&&)} the booleans of a list}
Supposing
\begin{lstlisting}[style=haskell,basicstyle=\tiny\ttfamily,mathescape]
list = `(&&)` True (`(&&)` True (`(&&)` False (`(&&)` True `True`)))
\end{lstlisting}
\end{block}
\begin{lstlisting}[style=haskell,basicstyle=\scriptsize\ttfamily,mathescape]
conjunct list = foldr (&&) True list
conjunct = foldr (&&) True
\end{lstlisting}
\end{frame}

%

\begin{frame}[fragile]
\frametitle{foldr}
\begin{block}{right folds replace constructors}
Let's append two lists
\end{block}
\end{frame}

\begin{frame}[fragile]
\frametitle{foldr}
\begin{block}{append two lists}
Supposing 
\begin{lstlisting}[style=haskell,basicstyle=\tiny\ttfamily,mathescape]
list1 = Cons A (Cons B (Cons C (Cons D Nil)))
list2 = Cons E (Cons F (Cons G (Cons H Nil)))
\end{lstlisting}
\end{block}
\end{frame}

\begin{frame}[fragile]
\frametitle{foldr}
\begin{block}{append two lists}
Supposing 
\begin{lstlisting}[style=haskell,basicstyle=\tiny\ttfamily,mathescape]
list1 = `Cons` A (`Cons` B (`Cons` C (`Cons` D `Nil`)))
list2 = Cons E (Cons F (Cons G (Cons H Nil)))
\end{lstlisting}
\end{block}
\begin{center}
\LARGE
?
\end{center}
\end{frame}

\begin{frame}[fragile]
\frametitle{foldr}
\begin{block}{append two lists}
\begin{itemize}
\item let \lstinline{`Cons` = Cons}
\item let \lstinline{`Nil`  = list2}
\end{itemize}
\end{block}
\end{frame}

\begin{frame}[fragile]
\frametitle{foldr}
\begin{block}{append two lists}
Supposing
\begin{lstlisting}[style=haskell,basicstyle=\tiny\ttfamily,mathescape]
list1 = `Cons` A (`Cons` B (`Cons` C (`Cons` D `list2`)))
list2 = Cons E (Cons F (Cons G (Cons H Nil)))
\end{lstlisting}
\end{block}
\begin{lstlisting}[style=haskell,basicstyle=\scriptsize\ttfamily,mathescape]
append list1 list2 = foldr Cons list2 list1
append = flip (foldr Cons)
\end{lstlisting}
\end{frame}

%

\begin{frame}[fragile]
\frametitle{foldr}
\begin{block}{right folds replace constructors}
Let's map a function on a list
\end{block}
\end{frame}

\begin{frame}[fragile]
\frametitle{foldr}
\begin{block}{map a function \lstinline{(f)} on a list}
Supposing 
\begin{lstlisting}[style=haskell,basicstyle=\tiny\ttfamily,mathescape]
list = Cons A (Cons B (Cons C (Cons D Nil)))
\end{lstlisting}
\end{block}
\end{frame}

\begin{frame}[fragile]
\frametitle{foldr}
\begin{block}{map a function \lstinline{(f)} on a list}
Supposing 
\begin{lstlisting}[style=haskell,basicstyle=\tiny\ttfamily,mathescape]
list = `Cons` A (`Cons` B (`Cons` C (`Cons` D `Nil`)))
\end{lstlisting}
\end{block}
\begin{center}
\LARGE
?
\end{center}
\end{frame}

\begin{frame}[fragile]
\frametitle{foldr}
\begin{block}{map a function \lstinline{(f)} on a list}
\begin{itemize}
\item let \lstinline{`Cons` = \x -> Cons (f x)}
\item let \lstinline{`Nil`  = Nil}
\end{itemize}
\end{block}
\end{frame}

\begin{frame}[fragile]
\frametitle{foldr}
\begin{block}{map a function \lstinline{(f)} on a list}
Supposing
\begin{lstlisting}[style=haskell,basicstyle=\tiny\ttfamily,mathescape]
consf x = Cons (f x)

list = `consf` A (`consf` B (`consf` C (`consf` D `Nil`)))
\end{lstlisting}
\end{block}
\begin{lstlisting}[style=haskell,basicstyle=\scriptsize\ttfamily,mathescape]
map f list = foldr (\x -> Cons (f x)) Nil list
map f = foldr (Cons . f) Nil
\end{lstlisting}
\end{frame}

%

\begin{frame}[fragile]
\frametitle{foldr}
\begin{block}{right folds replace constructors}
Let's append a list of lists
\end{block}
\end{frame}

\begin{frame}[fragile]
\frametitle{foldr}
\begin{block}{flatten a list of lists}
Supposing 
\begin{lstlisting}[style=haskell,basicstyle=\tiny\ttfamily,mathescape]
list = Cons lista (Cons listb (Cons listc (Cons listd Nil)))
\end{lstlisting}
\end{block}
\end{frame}

\begin{frame}[fragile]
\frametitle{foldr}
\begin{block}{flatten a list of lists}
Supposing 
\begin{lstlisting}[style=haskell,basicstyle=\tiny\ttfamily,mathescape]
list = `Cons` lista (`Cons` listb (`Cons` listc (`Cons` listd `Nil`)))
\end{lstlisting}
\end{block}
\begin{center}
\LARGE
?
\end{center}
\end{frame}

\begin{frame}[fragile]
\frametitle{foldr}
\begin{block}{flatten a list of lists}
\begin{itemize}
\item let \lstinline{`Cons` = append}
\item let \lstinline{`Nil`  = Nil}
\end{itemize}
\end{block}
\end{frame}

\begin{frame}[fragile]
\frametitle{foldr}
\begin{block}{flatten a list of lists}
Supposing
\begin{lstlisting}[style=haskell,basicstyle=\tiny\ttfamily,mathescape]
list = `append` lista (`append` listb (`append` listc (`append` listd `Nil`)))
\end{lstlisting}
\end{block}
\begin{lstlisting}[style=haskell,basicstyle=\scriptsize\ttfamily,mathescape]
flatten list = foldr append Nil list
flatten = foldr append Nil
\end{lstlisting}
\end{frame}

%

\begin{frame}[fragile]
\frametitle{foldr}
\begin{block}{right folds replace constructors}
Let's filter a list on predicate
\end{block}
\end{frame}

\begin{frame}[fragile]
\frametitle{foldr}
\begin{block}{filter a list on predicate \lstinline{(p)}}
Supposing 
\begin{lstlisting}[style=haskell,basicstyle=\tiny\ttfamily,mathescape]
list = Cons A (Cons B (Cons C (Cons D Nil)))
\end{lstlisting}
\end{block}
\end{frame}

\begin{frame}[fragile]
\frametitle{foldr}
\begin{block}{filter a list on predicate \lstinline{(p)}}
Supposing 
\begin{lstlisting}[style=haskell,basicstyle=\tiny\ttfamily,mathescape]
list = `Cons` A (`Cons` B (`Cons` C (`Cons` D `Nil`)))
\end{lstlisting}
\end{block}
\begin{center}
\LARGE
?
\end{center}
\end{frame}

\begin{frame}[fragile]
\frametitle{foldr}
\begin{block}{filter a list on predicate \lstinline{(p)}}
\begin{itemize}
\item let \lstinline{`Cons` = \x -> if p x then Cons x else id}
\item let \lstinline{`Nil`  = Nil}
\end{itemize}
\end{block}
\end{frame}

\begin{frame}[fragile]
\frametitle{foldr}
\begin{block}{filter a list on predicate \lstinline{(p)}}
Supposing
\begin{lstlisting}[style=haskell,basicstyle=\tiny\ttfamily,mathescape]
applyp x = if p x then Cons x else id

list = `applyp` A (`applyp` B (`applyp` C (`applyp` D `Nil`)))
\end{lstlisting}
\end{block}
\begin{lstlisting}[style=haskell,basicstyle=\scriptsize\ttfamily,mathescape]
filter p list = foldr (\x -> if p x then Cons x else id) Nil list
filter p = foldr (\x -> if p x then Cons x else id) Nil
filter p = foldr (\x -> bool id (Cons x) (p x)) Nil
filter p = foldr (bool id . Cons <*> p) Nil
\end{lstlisting}
\end{frame}

%

\begin{frame}[fragile]
\frametitle{foldr}
\begin{block}{right folds replace constructors}
Let's get the head of a list, or default for no head

\lstinline{:: a -> List a -> a}
\end{block}
\end{frame}

\begin{frame}[fragile]
\frametitle{foldr}
\begin{block}{the head of a list, or default for no head}
Supposing 
\begin{lstlisting}[style=haskell,basicstyle=\tiny\ttfamily,mathescape]
list = Cons A (Cons B (Cons C (Cons D Nil)))
\end{lstlisting}
\end{block}
\end{frame}

\begin{frame}[fragile]
\frametitle{foldr}
\begin{block}{the head of a list, or default for no head}
Supposing 
\begin{lstlisting}[style=haskell,basicstyle=\tiny\ttfamily,mathescape]
list = `Cons` A (`Cons` B (`Cons` C (`Cons` D `Nil`)))
\end{lstlisting}
\end{block}
\begin{center}
\LARGE
?
\end{center}
\end{frame}

\begin{frame}[fragile]
\frametitle{foldr}
\begin{block}{the head of a list, or default for no head}
\begin{itemize}
\item let \lstinline{`Cons` = \x _ -> x}
\item let \lstinline{`Nil`  = thedefault}
\end{itemize}
\end{block}
\end{frame}

\begin{frame}[fragile]
\frametitle{foldr}
\begin{block}{the head of a list, or default for no head}
Supposing
\begin{lstlisting}[style=haskell,basicstyle=\tiny\ttfamily,mathescape]
constant x _ = x

list = `constant` A (`constant` B (`constant` C (`constant` D `thedefault`)))
\end{lstlisting}
\end{block}
\begin{lstlisting}[style=haskell,basicstyle=\scriptsize\ttfamily,mathescape]
heador thedefault list = foldr constant thedefault list
heador thedefault = foldr constant thedefault
heador = foldr constant
\end{lstlisting}
\end{frame}

%

\begin{frame}[fragile]
\frametitle{foldr}
\begin{block}{right folds replace constructors}
Let's sequence a list of effects \lstinline{(f a)} and produce an effect (f) of list

\lstinline{:: Monad f => List (f a) -> f (List a)}
\end{block}
\end{frame}

\begin{frame}[fragile]
\frametitle{foldr}
\begin{block}{list of effects \lstinline{(f a)} to effect (f) of list}
Supposing 
\begin{lstlisting}[style=haskell,basicstyle=\tiny\ttfamily,mathescape]
list = Cons A (Cons B (Cons C (Cons D Nil)))
\end{lstlisting}
\end{block}
\end{frame}

\begin{frame}[fragile]
\frametitle{foldr}
\begin{block}{list of effects \lstinline{(f a)} to effect (f) of list}
Supposing 
\begin{lstlisting}[style=haskell,basicstyle=\tiny\ttfamily,mathescape]
list = `Cons` A (`Cons` B (`Cons` C (`Cons` D `Nil`)))
\end{lstlisting}
\end{block}
\begin{center}
\LARGE
?
\end{center}
\end{frame}

\begin{frame}[fragile]
\frametitle{foldr}
\begin{block}{list of effects \lstinline{(f a)} to effect (f) of list}
\begin{itemize}
\item let \lstinline[mathescape]{`Cons` = \a b -> do $\texttt{\{}$ x <- a; y <- b; return (Cons x y) $\texttt{\}}$ }
\item let \lstinline{`Nil`  = return Nil}
\end{itemize}
\end{block}
\end{frame}

\begin{frame}[fragile]
\frametitle{foldr}
\begin{block}{list of effects \lstinline{(f a)} to effect (f) of list}
Supposing
\begin{lstlisting}[style=haskell,basicstyle=\tiny\ttfamily,mathescape]
lift2cons a b = do { x <- a; y <- b; return (Cons a b)}

list = `lift2cons` A (`lift2cons` B (`lift2cons` C (`lift2cons` D `return Nil`)))
\end{lstlisting}
\end{block}
\begin{lstlisting}[style=haskell,basicstyle=\scriptsize\ttfamily,mathescape]
sequence list = foldr (lift2cons) (return Nil) list
sequence = foldr (lift2cons) (return Nil)
\end{lstlisting}
\end{frame}

\begin{frame}[fragile]
\frametitle{foldr}
\begin{block}{Observations}
\begin{itemize}
\item \lstinline[basicstyle=\ttfamily]$foldr$ may work on an infinite list.
  \begin{itemize}
  \item There is no \emph{order} specified, however, there is associativity.
  \item Depends on the strictness of the given function.
  \item Replaces the \lstinline[basicstyle=\ttfamily]$Nil$ constructor \emph{if it ever comes to exist}.
  \end{itemize}
\item The expression \lstinline[basicstyle=\ttfamily]$foldr Cons Nil$ leaves the list \emph{unchanged}.
  \begin{itemize}
  \item In other words, passing the list constructors to \lstinline[basicstyle=\ttfamily]$foldr$ produces an \emph{identity} function.
  \item A function that produces an identity, given constructors for a data type, is called its \emph{catamorphism}.
  \item \lstinline[basicstyle=\ttfamily]$foldr$ is the list catamorphism.
  \end{itemize}
\end{itemize}
\end{block}
\end{frame}
